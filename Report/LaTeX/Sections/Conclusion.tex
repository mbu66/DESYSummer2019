The efficiency of the reconstruction of the semileptonic decay of a W pair produced in an ${e}_{L}^{-}{e}_{R}^{+}$ collision does have an angular dependence that is not currently modelled in the Electroweak Polarisation fit for the ILC. The theta coordinate of the negatively charged W boson, ${\theta}_{{W}^{-}}$, is statistically limited in directions opposite to the initial electron's trajectory and so it is hard to observe the functional form of the efficiency dependance. The efficiency has a clear dependance on the theta coordinate of the lepton in the center of mass frame of the W it decayed from, ${\theta}_{l}^{*}$. There is a drop in efficiency as the leptons direction aligns with the beam pipe. There is no angular dependence observed in the efficiency as the phi coordinate of the lepton in the center of mass frame of the W it decayed from, ${\phi}_{l}^{*}$, is varied.
\\\\
It is also observed that the initial state radiation (ISR) does effect the efficiency of the reconstruction. For events with ISR energies below 1 GeV the reconstruction efficiency is higher by a constant factor, with a similar functional form. As the Electroweak Polarisation fit does not currently model ISR this is an important feature.
\\\\
A significant drop in the efficiency of the reconstruction is due to the beam background removal and due to events where there is no reconstructed isolated lepton, otherwise this reconstruction is seen to perform similarly, and slightly better, than previous studies.
\\\\
In future studies, the potential bias introduced by implementing the three ISR energy solutions in the reconstruction should be looked into and the appropriate cuts made. An alternative approach would be to perform a full kinematic fit to arrive at the best solution for the 4-momenta of the particles in the final state.

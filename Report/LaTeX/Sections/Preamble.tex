% TODO Clean this up

\documentclass[12pt,oneside,notitlepage,abstracton,a4paper]{article}


% TODO: Find out what these are for
\usepackage{epsfig,scrpage2}
\usepackage{amsmath}
\usepackage{makecell}
\usepackage{graphicx}
\usepackage{booktabs}
\usepackage{arydshln}
\usepackage{setspace} % For singlespace
\usepackage[utf8]{inputenc}
\usepackage[T1]{fontenc}
\usepackage{tablefootnote}

\usepackage{amssymb}  % Equations
\usepackage{xcolor}   % Colours
\usepackage{listings} % Code listings
\usepackage{cancel,xcolor}
\newcommand{\Cancel}[2][black]{{\color{#1}\cancel{\color{black}#2}}}


%%%%%%%%%%%%%%%%%%%%%%%%%%%%%%%%%%%%%%%%%%%%%%%%%%%%%%%%%%%%%%%%%%%%%%%%%%%%%%%%
% Document layout
%%%%%%%%%%%%%%%%%%%%%%%%%%%%%%%%%%%%%%%%%%%%%%%%%%%%%%%%%%%%%%%%%%%%%%%%%%%%%%%%
% TODO Put some order into this part

% For Titlepage
\renewcommand{\headfont}{\normalfont}
  % \cfoot{\pagemark}
  \date{\normalsize \today}
\newcommand{\hwidthline}{
  \noindent\makebox[\linewidth]{\rule{\linewidth}{0.4pt}}
}
%-------------------------------------------------------------------------------
% For TODOS:
\newenvironment{TODOlist}{%
    \begin{list}{\textcolor{DESYorange}{!}\textcolor{DESYcyan}{$\bullet$}\textcolor{DESYorange}{!}}{\setlength{\itemsep}{0.5pt}}
  }{%
    \end{list}%
}

%-------------------------------------------------------------------------------
% To avoid float-only pages change minimum content on page that is necessary
% for page to become float-only
\renewcommand{\floatpagefraction}{.7}

%-------------------------------------------------------------------------------
% For manipulating counters (e.g. figure counter in appendix)
\usepackage{chngcntr}

%-------------------------------------------------------------------------------
% In appendix use the section letter as first numbering letter for everything
\newcommand{\setappendixnumbering}{%
  \renewcommand{\thesection}{\Alph{section}}
  \renewcommand{\theequation}{\thesection.\arabic{equation}}
  \renewcommand{\thefigure}{\thesection.\arabic{figure}}
  \renewcommand{\thetable}{\thesection.\arabic{table}}
  \counterwithin{equation}{section}
  \counterwithin{figure}{section}
  \counterwithin{table}{section}
}

%-------------------------------------------------------------------------------
% Never break footnote onto multiple pages!
\interfootnotelinepenalty=10000

%-------------------------------------------------------------------------------
% Prevent floats to appear below footnotes
\usepackage[bottom]{footmisc}

%-------------------------------------------------------------------------------
% Remove section numbering
% \makeatletter
%   \renewcommand\@seccntformat[1]{}
% \makeatother

%-------------------------------------------------------------------------------
% Add line numbering
\iflinenumbers
  \usepackage{lineno}
  \linenumbers
\fi

%-------------------------------------------------------------------------------
% For affiliations
\usepackage[affil-it]{authblk}

%-------------------------------------------------------------------------------
% Title page layout (title, authors, affil)
\usepackage{etoolbox} % Tool to modify things (not sure, also need to sort somewhere better...)
\makeatletter % Change Title
\patchcmd{\@maketitle}{\LARGE \@title}{\fontsize{16}{19.2}\selectfont\@title}{}{}
\makeatother
\renewcommand\Authfont{\fontsize{12}{14.4}\selectfont}  % Change author
\renewcommand\Affilfont{\fontsize{9}{10.8}\itshape}     % Change affiliation

%-------------------------------------------------------------------------------
% Smaller side margins
\usepackage{geometry}
\geometry{
 total={170mm,257mm},
 left=20mm,
 top=20mm,
 }

%%%%%%%%%%%%%%%%%%%%%%%%%%%%%%%%%%%%%%%%%%%%%%%%%%%%%%%%%%%%%%%%%%%%%%%%%%%%%%%%
% Directory setup
%%%%%%%%%%%%%%%%%%%%%%%%%%%%%%%%%%%%%%%%%%%%%%%%%%%%%%%%%%%%%%%%%%%%%%%%%%%%%%%%

% Custom paths
\newcommand{\imagepath}{/Users/mbk/Documents/LATEX/SummerStudentReportTemplate-master/Plots/high_stat}
\newcommand{\logopath}{../Logos}
\newcommand{\feynmanpath}{./FeynmanDiagrams}


%%%%%%%%%%%%%%%%%%%%%%%%%%%%%%%%%%%%%%%%%%%%%%%%%%%%%%%%%%%%%%%%%%%%%%%%%%%%%%%%
% Graphics, Figures, Tables
%%%%%%%%%%%%%%%%%%%%%%%%%%%%%%%%%%%%%%%%%%%%%%%%%%%%%%%%%%%%%%%%%%%%%%%%%%%%%%%%

% Custom colors
\usepackage{xcolor}
\definecolor{DESYcyan}{RGB}{0,166,235}
\definecolor{DESYorange}{RGB}{242,142,0}
\definecolor{DESYgray}{RGB}{119,119,119}
\definecolor{bjetpurple}{RGB}{117,112,179}

%-------------------------------------------------------------------------------
% Feynman and other diagrams
\usepackage[compat=1.1.0]{tikz-feynman}
\usepackage{contour}
\usetikzlibrary{arrows,shapes,positioning}
\usetikzlibrary{decorations.markings}
\usetikzlibrary{decorations.pathmorphing}
\usetikzlibrary{decorations.pathreplacing}
\usetikzlibrary{patterns}
\usetikzlibrary{plotmarks}
\usetikzlibrary{shadows}

%-------------------------------------------------------------------------------
% Pretty captions and subcaptions
\usepackage[margin=8mm,font=small,labelfont=bf,format=plain]{caption}
\usepackage[margin=8mm,font=small,labelfont=bf,format=plain]{subcaption}

%-------------------------------------------------------------------------------
% Table top captions and alignment
\captionsetup[table]{position=top}
\captionsetup[subtable]{position=top}

%-------------------------------------------------------------------------------
% Nice features for tables
\usepackage{multirow} % Cells that span multiple rows
\newcommand{\vcell}[3]{\parbox[t]{#1}{\multirow{#2}{*}{\rotatebox[origin=c]{90}{#3}}}}

%-------------------------------------------------------------------------------
% Allows fixing float positions with the [H] option
% \usepackage{float}

%-------------------------------------------------------------------------------
% Shortcut for referencing a subfigure within a figure caption
\newcommand{\subfigref}[1]{({\protect\subref{#1}})}

%-------------------------------------------------------------------------------
% Force floats to appear in its section
% \usepackage[section]{placeins}

%%%%%%%%%%%%%%%%%%%%%%%%%%%%%%%%%%%%%%%%%%%%%%%%%%%%%%%%%%%%%%%%%%%%%%%%%%%%%%%%
% Text manipulation
%%%%%%%%%%%%%%%%%%%%%%%%%%%%%%%%%%%%%%%%%%%%%%%%%%%%%%%%%%%%%%%%%%%%%%%%%%%%%%%%

% Hyperlinks in pdf
\usepackage[
  bookmarks,                   %% PDF-Lesezeichen
  bookmarksopenlevel=1,        %% ...um eine Ebene
  bookmarksnumbered=true,      %% Lesezeichen numerieren
  pdfstartpage={1},            %% mit welcher Seite das PDF öffnen
  pdfstartview={FitH},         %% Zoom auf Seitenbreite
  pdfkeywords={},              %% Stichwörter fürs PDF, kommagetrennt
  pdfsubject={},               %% Themenbeschreibung kurz
  pdfcreator={LaTeX with KOMA-Script and hyperref package},
  hyperfootnotes=true,         %% Links auf Fußnoten
  linkbordercolor={0 1 1},     %% Rahmenfarbe interne Links
  menubordercolor={0 1 1},     %% Rahmenfarbe Literaturlinks
  urlbordercolor={1 0 0}       %% Rahmenfarbe externe Links
]{hyperref}                    %% Hyperlinks und Lesezeichen in PDF



%%%%%%%%%%%%%%%%%%%%%%%%%%%%%%%%%%%%%%%%%%%%%%%%%%%%%%%%%%%%%%%%%%%%%%%%%%%%%%%%
% Additions for math
%%%%%%%%%%%%%%%%%%%%%%%%%%%%%%%%%%%%%%%%%%%%%%%%%%%%%%%%%%%%%%%%%%%%%%%%%%%%%%%%

% Symbols with text above/below
\newcommand{\textover}[2]{\stackrel{\text{#2}}{#1}}
\newcommand{\eqtext}[1]{\textover{=}{#1}} %Equal sign with text over it

%-------------------------------------------------------------------------------
% Dirac notation
\usepackage{slashed}

%-------------------------------------------------------------------------------
% New fancy command for column vector with any number of rows
% (Taken from https://tex.stackexchange.com/questions/2705/typesetting-column-vector)
\newcount\colveccount
\newcommand*\colvec[1]{
        \global\colveccount#1
        \begin{pmatrix}
        \colvecnext
}
\def\colvecnext#1{
        #1
        \global\advance\colveccount-1
        \ifnum\colveccount>0
                \\
                \expandafter\colvecnext
        \else
                \end{pmatrix}
        \fi
}

%%%%%%%%%%%%%%%%%%%%%%%%%%%%%%%%%%%%%%%%%%%%%%%%%%%%%%%%%%%%%%%%%%%%%%%%%%%%%%%%
% Code
%%%%%%%%%%%%%%%%%%%%%%%%%%%%%%%%%%%%%%%%%%%%%%%%%%%%%%%%%%%%%%%%%%%%%%%%%%%%%%%%
% For code snipplet inputs
\usepackage{lmodern}
\usepackage{listings}
\lstset{
  language=[90]Fortran,
  basicstyle=\footnotesize,        % the size of the fonts that are used for the code
  keywordstyle=\color{red},
  commentstyle=\color{green},
  morecomment=[l]{!\ }% Comment only with space after !
  breaklines=true,                 % sets automatic line breaking
  frame=single,	                   % adds a frame around the code
  keepspaces=true,                 % keeps spaces in text, useful for keeping indentation of code (possibly needs columns=flexible)
  % numbers=left,                    % where to put the line-numbers; possible values are (none, left, right)
  % numbersep=5pt,                   % how far the line-numbers are from the code
  % numberstyle=\tiny\color{mygray}, % the style that is used for the line-numbers
  % stepnumber=2,                    % the step between two line-numbers. If it's 1, each line will be numbered
  rulecolor=\color{black},         % if not set, the frame-color may be changed on line-breaks within not-black text (e.g. comments (green here))
  showspaces=false,                % show spaces everywhere adding particular underscores; it overrides 'showstringspaces'
  showstringspaces=false,          % underline spaces within strings only
  showtabs=false,                  % show tabs within strings adding particular underscores
  % tabsize=2,	                   % sets default tabsize to 2 spaces
}


%%%%%%%%%%%%%%%%%%%%%%%%%%%%%%%%%%%%%%%%%%%%%%%%%%%%%%%%%%%%%%%%%%%%%%%%%%%%%%%%
% Bibliography
%%%%%%%%%%%%%%%%%%%%%%%%%%%%%%%%%%%%%%%%%%%%%%%%%%%%%%%%%%%%%%%%%%%%%%%%%%%%%%%%
\usepackage[english]{babel}% Recommended
\usepackage{csquotes}% Recommended
\usepackage[style=numeric-comp,sorting=none]{biblatex}
\addbibresource{library.bib}% Syntax for version >= 1.2

\newcommand{\customprintbibliography}{%
  % \begingroup
  \printbibliography[heading=none]
  % \endgroup
  \newpage
}


%%%%%%%%%%%%%%%%%%%%%%%%%%%%%%%%%%%%%%%%%%%%%%%%%%%%%%%%%%%%%%%%%%%%%%%%%%%%%%%%
% Symbol shortcuts
%%%%%%%%%%%%%%%%%%%%%%%%%%%%%%%%%%%%%%%%%%%%%%%%%%%%%%%%%%%%%%%%%%%%%%%%%%%%%%%%

\usepackage{amssymb}

%-------------------------------------------------------------------------------
% Math/physics symbols
\newcommand{\Lagr}{\mathcal{L}}
\newcommand{\mbar}{\bar{m}}
\newcommand{\oper}{\mathcal{O}}
\newcommand{\DEG}{^{\circ}}

%-------------------------------------------------------------------------------
% Line stuff
\newcommand{\flend}{&&\\} % End of a flalign newenvironment line

%-------------------------------------------------------------------------------
% Particles
\newcommand{\eP}{e^{+}}
\newcommand{\eM}{e^{-}}
\newcommand{\lP}{l^{+}}
\newcommand{\lM}{l^{-}}

\newcommand{\fbar}{\bar{f}}

\newcommand{\eL}{\eM_{L}}
\newcommand{\eR}{\eM_{R}}
\newcommand{\pL}{\eP_{L}}
\newcommand{\pR}{\eP_{R}}

\newcommand{\nue}{\nu_e}
\newcommand{\numu}{\nu_{\mu}}
\newcommand{\nutau}{\nu_{\tau}}
\newcommand{\nul}{\nu_l}

\newcommand{\nubar}{\bar{\nu}}
\newcommand{\nuebar}{\bar{\nue}}
\newcommand{\nulbar}{\bar{\nul}}

\newcommand{\qbar}{\bar{q}}
\newcommand{\qu}{q_{u}}
\newcommand{\qd}{q_{d}}
\newcommand{\qubar}{\bar{q}_{u}}
\newcommand{\qdbar}{\bar{q}_{d}}

\newcommand{\tbar}{\bar{t}}

\newcommand{\vis}{\text{vis}}
\newcommand{\inv}{\text{inv}}

\newcommand{\WP}{W^{+}}
\newcommand{\WM}{W^{-}}

%%%%%%%%%%%%%%%%%%%%%%%%%%%%%%%%%%%%%%%%%%%%%%%%%%%%%%%%%%%%%%%%%%%%%%%%%%%%%%%%
%%%%%%%%%%%%%%%%%%%%%%%%%%%%%%%%%%%%%%%%%%%%%%%%%%%%%%%%%%%%%%%%%%%%%%%%%%%%%%%%










%TODO TO ORDER:
%%%%%%%%%%%%%%%%%%%%%%%%%%%%%%%%%%%%%%%%%%%%%%%%%%%%%%%%%%%%%%%%%%%%%%%%%%%%%%%%

% TODO What is this??
\setcounter{secnumdepth}{3}

\setlength{\parindent}{0em}
\setlength{\parskip}{0ex plus0.5ex minus0ex}
\pagestyle{scrheadings}

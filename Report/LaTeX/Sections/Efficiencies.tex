The following analysis took place using python scripts to analyse the data stored in the produced root file. This data was generated by choosing the best of all 3 ${E}_{\gamma}$ solutions, with  ${m}_{\gamma} = {m}_{\nu} = 0$. Unless otherwise explicitly stated, the beam background removal was not cheated but the lorentz boost into the centre of mass frame was performed.
\\\\
Before any analysis it was ensured that only the muon signal is considered, as this was the focus of this analysis, by making a cut on the events at generator level.

%---------------------------------------------------------------------------------------------------------------------------------------------------------%---------------------------------------------------------------------------------------------------------------------------------------------------------
\subsection{Cut Flow}
\label{SUBSEC:CutFlow}

All the same cuts as in Ivan’s thesis \cite{IvanMarchesini} were sequentially applied to obtain a cut flow histogram, which tests the efficiency of each of the cuts. This histogram was created by dividing the number of events left after each cut by the initial number of events considered as signal. The results of this analysis were then compared to Ivan's to see how it is performing. This was done for two different statistical sample sizes, the higher statistics sample being the sample used throughout the rest of this report. Further, the flow diagram was created again by cheating the beam background removal, and by applying the additional initial cut that the ISR energy is small (${E}_{\gamma}^{MC} < 1$ GeV). The final plot was made to see if the ISR made a considerable contribution to the efficiency of the cuts.
\\\\
There are a few subtle differences between the cuts that Ivan and the cuts made in this analysis, due to the reconstruction methods each implemented. The first was the cut on the jet reconstruction variables ${y}_{+}$ and ${y}_{-}$. In this analysis FastJets was forced to reconstruct two jets, corresponding to the 2 quarks, and physically this is the minimum number of jets that can occur, as single quark cannot be created. This means the cut on the ${y}_{-}$ variable that Ivan makes is unphysical in this analysis. Ivan applies this cut because he also considered final states with 4 quarks. The lepton cut that was made here make is also different to Ivan’s. The IsolatedLeptonTaggingProcessor was used, so the implemented cut was that a single isolated lepton is reconstructed. There are a considerable number of events where the processor reconstructed zero particles, and so this cut can be significant. Ivan on the other hand used a jet reconstruction processor with a specific energy cut to reconstruct isolated leptons. The final difference is that Ivan makes a 'charged lepton' cut, which is not described in his thesis. A similar cut is not made in this analysis and so the corresponding efficiency will not change across this cut.
\\\\
For convenience the cuts and efficiencies are tabulated in Table.~\ref{TAB:SelectionEfficiencies}, and the resulting total flow diagram can be seen in (Figure.~\ref{FIG:Flow}).
\\\\
\begin{table}[!]
    \centering
    \caption{
        Selection efficiency of sequantially applied cuts. Where the post ISR correction ${m}_{W}^{lep}$ was calculated using all 3 possible ${E}_{\gamma}$ solutions. (*) Indicates cuts where the current and Ivan's cuts differ, as discussed in the text.
    }
    \resizebox{0.8\textwidth}{!}{%
    \begin{tabular}{|l|l|l|l|l|l|} \hline
        Order & Cut description & \multicolumn{4}{c|}{Efficiency [\%]} \\ \cline{3-6}
        & & \multicolumn{3}{c|}{My Results} & Ivan's Results \\  \cline{3-5}
        & & n = 2129 & \multicolumn{2}{c|}{n = 99419} & n = 107233 \\ \cline{4-5}
        & & & no cheat & cheat & \\ \hline \hline
        0 & muon signal & 100.00 & 100.00 & 100.00 & 100.00 \\ \hline
        1 & track multiplicity\tablefootnote{track mulitplicity was taken as the number of reconstructed charged particles.} ${n}_{tracks} \ge 10$ & 97.13 & 97.01 & 96.23 & 99.996 \\ \hline
        2 & center of mass energy $\sqrt{s} > 100$ GeV & 92.29 & 91.69 & 84.35 & 97.96 \\ \hline
        3 & total transverse momentum ${P}_{T} > 5$ GeV & 91.16 & 90.47 & 83.28 & 96.69 \\ \hline
        4 & total energy ${E}_{SUM} < 500$ GeV & 89.66 & 89.28 & 82.70 & 95.36 \\ \hline
        5 & $\ln{({y}_{+})} \in [-12, -3]$ (*) & 88.69 & 88.08 & 82.47 & 95.01 \\ \hline
        6 & 1 lepton found (*) & 80.65 & 80.77 & 81.50 & 78.75 \\ \hline
        7 & pre ISR correction ${m}_{W}^{lep} \in [20, 250]$ GeV &  78.23 & 77.94 & 77.84 & 76.61 \\ \hline
        8 & tau discrimination\tablefootnote{${\tau}_{discr}$ defined by ${\tau}_{discr} = {(\frac{2{E}_{lep}}{\sqrt{s}})}^{2} + {(\frac{{m}_{W}^{lep}}{{m}_{W}^{true}})}^{2}$} &  76.05 & 75.60 & 75.73 & 74.07 \\ \hline
        9 & charged lepton (*) & 76.05 & 75.60 & 75.73 & 73.51 \\ \hline
        10 & isolation variable\tablefootnote{$\Delta{\Omega}_{iso}$ defined as,
        \begin{align}
            ({\phi}_{lep} - {\phi}_{had}) < \pi \to \Delta{\Omega}_{iso} &= \sqrt{{({\theta}_{lep} - {\theta}_{had})}^{2}+{({\phi}_{lep} - {\phi}_{had})}^{2}} \\
            ({\phi}_{lep} - {\phi}_{had}) \ge \pi \to \Delta{\Omega}_{iso} &= \sqrt{{({\theta}_{lep} - {\theta}_{had})}^{2} + {(2\pi - |{\phi}_{lep} - {\phi}_{had} |)}^{2}} \, .
        \end{align}} ${\Delta\Omega}_{iso} > 0.5$ & 76.01 & 75.58 & 75.72 & 73.42 \\ \hline
        11 & post ISR correction ${m}_{W}^{lep} \in [40, 120]$ GeV & 72.90 & 72.77 & 72.33 & 70.13 \\ \hline
        12 & post ISR correction ${m}_{W}^{had} \in [40, 120]$ GeV & 63.21 & 62.92 & 70.52 & 66.93 \\ \hline
        13 & $\cos{{\theta}_{W}} > -0.95$ & 63.02 & 62.65 & 70.21 & 66.78 \\ \hline
        \end{tabular}
        }
        \label{TAB:SelectionEfficiencies}
    \end{table}

\begin{figure}[!]
    \centering
    \includegraphics[width=0.8\textwidth]{\imagepath/P_HistFlow3_All.pdf}
    \caption{
    Histograms showing the cut flows for Ivan's results, my results (with and whithout cheating overlay removal), and my results again when the ISR photon has low energy (${E}_{\gamma} < 1$). The cuts on the x axis are as defined in Table.~\ref{TAB:SelectionEfficiencies}.
    }
    \label{FIG:Flow}
\end{figure}

There is a fairly large difference in the cut efficiencies before the 'lepton' cut. The cut on finding only one isolated lepton would make a considerable difference to the performace of the reconstruction. In all of the events where no isolated leptons were reconstructed the leptonic W boson is reconstructed entiely from the invisble neutrino, which will be completely wrong. We see that 'My Results' and the 'Cheated Beam Background' converge on this lepton cut and It is believed this is because the differences prior were caused by such events. When this lepton cut is applied to the current results before the rest of the cuts the large discrepancy is not observed, providing support to this hypothesis.
\\\\
After the lepton cut, it can be sees that the results of this analysis are generally perform better than Ivan's, untill the cut on the ${m}_{W}^{had}$ where it is considerably worse. By looking at the cheated results though, which don't drop in efficiency as dramatically, it can be suggested that this is due to the performsnce of the beam background removal process. Further, after the lepton cut the ${E}_{\gamma}^{MC} < 1$ GeV signal is consistently performing better than the full signal, which suggests that large ISR energies reduces the efficiency of the reconstruciton and so the detector.

%---------------------------------------------------------------------------------------------------------------------------------------------------------%---------------------------------------------------------------------------------------------------------------------------------------------------------
\subsection{Angle Cut Efficiencies}
\label{SUBSEC:AngleCutEfficiencies}
The final component of this analysis, to evaluate the angular dependancy of the cut efficiency, was then performed. It was done by dividing an angular distribution histogram with all of the perviously defined cuts applied, by the same histogram where the only applied cut is that it was a muon signal. This was done for both the full signal and the ${E}_{\gamma}^{MC} < 1$ GeV signal, the results of which are shown in Figure.~\ref{FIG:AngleEfficiencies}.
\\
\begin{figure}
    \begin{subfigure}[t]{0.32\textwidth}
        \includegraphics[width=\textwidth]{\imagepath/ThetaMin_cos.pdf}
        \caption{}
        \label{SUBFIG:ThetaMin}
    \end{subfigure}
    \begin{subfigure}[t]{0.32\textwidth}
        \includegraphics[width=\textwidth]{\imagepath/ThetaLep_cos.pdf}
        \caption{}
        \label{SUBFIG:ThetaLep}
    \end{subfigure}
    \begin{subfigure}[t]{0.32\textwidth}
        \includegraphics[width=\textwidth]{\imagepath/PhiLep.pdf}
        \caption{}
        \label{SUBFIG:PhiLep}
    \end{subfigure}\\
    \begin{subfigure}[t]{0.32\textwidth}
        \includegraphics[width=\textwidth]{\imagepath/P_E_ThetaMin_cos_err_Both.pdf}
        \caption{}
        \label{SUBFIG:ThetaMinError}
    \end{subfigure}
    \begin{subfigure}[t]{0.32\textwidth}
        \includegraphics[width=\textwidth]{\imagepath/P_E_ThetaLep_cos_err_Both.pdf}
        \caption{}
        \label{SUBFIG:ThetaLepError}
    \end{subfigure}
    \begin{subfigure}[t]{0.32\textwidth}
        \includegraphics[width=\textwidth]{\imagepath/P_E_PhiLep_err_Both.pdf}
        \caption{}
        \label{SUBFIG:PhiLepError}
    \end{subfigure}
    \caption{
    \subfigref{SUBFIG:ThetaMin}, \subfigref{SUBFIG:ThetaLep} and \subfigref{SUBFIG:PhiLep} are the extracted angles as defined in Figure.~\ref{FIG:Angles}. \\
    \subfigref{SUBFIG:ThetaMinError}, \subfigref{SUBFIG:ThetaLepError} and \subfigref{SUBFIG:PhiLepError} are the associated cut efficiencies, after applying the cuts outlined in Table.~\ref{TAB:SelectionEfficiencies}.
    }
    \label{FIG:AngleEfficiencies}
\end{figure}

The $\cos{({\theta}_{{W}^{-}})}$ plot can be seen to be fairly statistically limited towards -1, in the backwards direction, and this greatly reduces the efficiency here, it is however reasonably constant throughout the rest of the distribution. The $\cos{({\theta}_{l}^{*})}$ has a clear angular dependance, with poor efficiency in regions closely aligned to the beam-pipe. Due to the coordinate system used this is not trivially explained, however for W bosons highly aligned to the beam pipe, the lepton appears to decay in a preferably transverse direction in the center of mass frame of such a boson. ${\phi}_{l}^{*}$ has a uniform efficiency as well as angular distribution, somewhat expected due to the uniformity of the W bosons ${\phi}$ distribution. This may not be the case if the stared (*) coordinate system was oriented such that ${\phi}_{l}^{*} = 0$ lied in the plane defined by the beam axis and the W boson, but that has not been explored in this report.
\\\\
The ${E}_{\gamma}^{MC} < 1$ GeV signal performed very similarly to the full signal, but the magnitude of the efficiency seemed to be different by a reasonably consistent factor.

Analysis of the pair production of W bosons in an electron positron collider (see Figure.~\ref{FEY:SemileptonicDecays}) is an important tool for probing the chiral nature of the electroweak interaction for physics Beyond the Standard Model (BSM). A particles chirality is determined by whether the particle transforms in a right- or left-handed representation of the Poincaré group. For massless particles, which travel at the speed of light in all frames, this is equivalent to its helicity which is defined by the projection of the particles spin vector onto its momentum vector. If this projection is positive the helicity is right-handed, and if it is negative the particle is left-handed. Chirality is a bit more subtle for massive particles, because the direction of their momentum is not frame invariant. The charged W bosons only couple to left handed particles and right handed anti-particles, and so the t-channel W pair production mode is only active if the initial state consists of a left-handed electron and right-handed positron (${e}_{L}^{-}{e}_{R}^{+}$). The s-channel, however, is active as long as the electron and position have opposite chirality. This means by adjusting the ratio of ${e}_{L}^{-}$ and ${e}_{R}^{+}$ in the particle beams of a collider, it is possible to switch the t-channel W pair production on and off. This can be used to probe the chiral structure of the electroweak interaction. Furthermore, the triple gauge vertex in the s-channel is a direct consequence of the chiral nature of the electroweak interaction and understanding this W pair production will help us better probe this vertex in a hunt for BSM effects.
\\
\begin{figure}
  \centering
  \begin{tikzpicture}
  \begin{feynman}[scale=1] % using the vertex in brackets () allows fixing of vertex
    \vertex (m) at (-1, 0);
    \vertex (n) at (0.5, 0);

    \vertex (r) at (-1.7,-0.5);
    \vertex (s) at (-1.7, 0.5);

    \vertex (a) at (-1.5,-1);s
    \vertex (b) at ( 1.75,-1) ;
    \vertex (c) at (-1.5, 1);
    \vertex (d) at ( 1.75, 1) ;

    \vertex (u) at ( 3.5,-1.5) {$l^-$};
    \vertex (v) at ( 3.5,-0.5) {$\bar{\nu}_l$};

    \vertex (q1) at ( 3.5, 1.5) {$q$};
    \vertex (q2) at ( 3.5, 0.5) {$\bar{q'}$};

    \vertex (i1) at (-3,-1) {$e^-$};
    \vertex (i2) at (-3, 1) {$e^+$};

    \vertex (p1) at (0,-1.5) {$\gamma$};
    \vertex (p2) at (0, 1.5) {$\gamma$};

    \diagram* {
      (i1) -- [fermion] (r) --[fermion] (m) ,
      (i2) -- [anti fermion] (s) -- [anti fermion] (m) ,
      (m) -- [photon, edge label=$Z/\gamma$] (n),
      (b) -- [photon, edge label=$W^-$, swap] (n),
      (d) -- [photon, edge label=$W^+$] (n),
      (b) -- [fermion] (u),
      (b) -- [anti fermion] (v),
      (d) -- [fermion] (q1),
      (d) -- [anti fermion] (q2),
      (r) -- [photon] (p1),
      (s) -- [photon] (p2),
    };
  \end{feynman}
\end{tikzpicture}

  \begin{tikzpicture}
  \begin{feynman}[scale=1] % using the vertex in brackets () allows fixing of vertex
    \vertex (n) at (0, 0.75) ;
    \vertex (m) at (0, -0.75) ;

    \vertex (r) at (-1.2,-1);
    \vertex (s) at (-1.2, 1);

    \vertex (a) at (-1,-1) ;
    \vertex (b) at ( 1.25,-1) ;
    \vertex (c) at (-1, 1);
    \vertex (d) at ( 1.25, 1) ;

    \vertex (u) at ( 3,-1.5) {$l^-$};
    \vertex (v) at ( 3,-0.5) {$\bar{\nu}_l$};

    \vertex (q1) at ( 3, 1.5) {$q$};
    \vertex (q2) at ( 3, 0.5) {$\bar{q'}$};

    \vertex (i1) at (-2.5,-1) {$e_L^-$};
    \vertex (i2) at (-2.5, 1) {$e_R^+$};

    \vertex (p1) at (0,-1.5) {$\gamma$};
    \vertex (p2) at (0, 1.5) {$\gamma$};


    \diagram* {
      (i1) -- [fermion] (r) --[fermion] (m) -- [fermion, edge label=$\nu_e$] (n),
      (i2) -- [anti fermion] (s) -- [anti fermion] (n),
      (b) -- [photon, edge label=$W^-$, swap] (m),
      (d) -- [photon, edge label=$W^+$] (n),
      (b) -- [fermion] (u),
      (b) -- [anti fermion] (v),
      (d) -- [fermion] (q1),
      (d) -- [anti fermion] (q2),
      (r) -- [photon] (p1),
      (s) -- [photon] (p2),

    };
  \end{feynman}
\end{tikzpicture}

  \caption{Lowest order Feynman diagram of semileptonic W pair decay in the $\mu\nubar q\qbar'$ final state at $\eP\eM$ colliders, with two ISR photons emitted. A s-channel (left) and t-channel (right) diagram are shown. Alternately the ${W}^{+}$ could decay leptonically and the ${W}^{-}$ hadronically, which would result in a $\mu^{+}\nu q\qbar'$ final state (not drawn).}
  \label{FEY:SemileptonicDecays}
\end{figure}

The data analysed in this report contains events where an initial state ${e}_{L}^{-}{e}_{R}^{+}$, with a center of mass energy of 500 GeV, produces a pair of W bosons, which in turn decay semileptonically. A semileptonic decay is defined such that one of the W bosons decays leptonically into a lepton and a neutrino, and the other boson decays hadronically into a pair of quarks. In this report, the muon signal of this decay is analysed. From the reconstruction of this interaction, the angular dependence of the detector and reconstruction efficiency is be analysed. This efficiency can then be implemented into the Electroweak Polarisation fit for the International Linear Collider (ILC), which currently assumes a global efficiency value of 60\%, and its effects analysed further.
\\\\\
In Section.~\ref{SEC:MyProcessor} the analysis methods used to extract the 4-momenta information of the final state particles will be discussed, along with a description of the angular distributions that are used. A discussion of how the efficiency of the reconstruction and its angular dependence are evaluated, and how they perform, is conducted in Section.~\ref{SEC:ApplyingCuts}. Finally, the findings of this report are concluded in Section.~\ref{SEC:Conclusion}.

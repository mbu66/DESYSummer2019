\begin{tikzpicture}
  \begin{feynman}[scale=1] % using the vertex in brackets () allows fixing of vertex
    \vertex (m) at (-1, 0);
    \vertex (n) at (0.5, 0);

    \vertex (r) at (-1.7,-0.5);
    \vertex (s) at (-1.7, 0.5);

    \vertex (a) at (-1.5,-1);s
    \vertex (b) at ( 1.75,-1) ;
    \vertex (c) at (-1.5, 1);
    \vertex (d) at ( 1.75, 1) ;

    \vertex (u) at ( 3.5,-1.5) {$l^-$};
    \vertex (v) at ( 3.5,-0.5) {$\bar{\nu}_l$};

    \vertex (q1) at ( 3.5, 1.5) {$q$};
    \vertex (q2) at ( 3.5, 0.5) {$\bar{q'}$};

    \vertex (i1) at (-3,-1) {$e^-$};
    \vertex (i2) at (-3, 1) {$e^+$};

    \vertex (p1) at (0,-1.5) {$\gamma$};
    \vertex (p2) at (0, 1.5) {$\gamma$};

    \diagram* {
      (i1) -- [fermion] (r) --[fermion] (m) ,
      (i2) -- [anti fermion] (s) -- [anti fermion] (m) ,
      (m) -- [photon, edge label=$Z/\gamma$] (n),
      (b) -- [photon, edge label=$W^-$, swap] (n),
      (d) -- [photon, edge label=$W^+$] (n),
      (b) -- [fermion] (u),
      (b) -- [anti fermion] (v),
      (d) -- [fermion] (q1),
      (d) -- [anti fermion] (q2),
      (r) -- [photon] (p1),
      (s) -- [photon] (p2),
    };
  \end{feynman}
\end{tikzpicture}

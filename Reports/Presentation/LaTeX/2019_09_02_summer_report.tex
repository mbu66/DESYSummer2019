%For DESY PC
\documentclass[xcolor=dvipsnames,compress,DESYcyan,9pt,unknownkeysallowed]{beamer}

\newcommand{\logopath}{../Logos}


\usepackage{blindtext}
% \usepackage[utf8,latin1]{inputenc}
\usepackage[T1]{fontenc}
\usepackage[english]{babel}
\usepackage{amsmath, amscd} %amscd for commutative diagrams
\usepackage{mathtools}
\usepackage{amssymb}
\usepackage{array}
\usepackage{blkarray}
\usepackage{color}
\usepackage{enumerate}
\usepackage{fancybox}
\usepackage{float}
\usepackage[hang]{footmisc}
\usepackage{graphicx}
\usepackage{listings}
\usepackage{lmodern}
\usepackage{multirow}
\usepackage{pifont}
\usepackage{rotating}
\usepackage[small, sl, SL,sf, SF, bf, hang, raggedright]{subfigure}
\usepackage{tabularx}
\usepackage{tcolorbox}
\usepackage{textpos}
\usepackage{textcomp}
\usepackage{tikz}
\usepackage{upgreek}
\usepackage{appendixnumberbeamer}
\usepackage{courier} % For programming style font

\definecolor{DESYcyan}{RGB}{0,166,235}
\definecolor{DESYorange}{RGB}{242,142,0}
\definecolor{DESYgray}{RGB}{119,119,119}
\definecolor{bjetpurple}{RGB}{117,112,179}

\usepackage[font=footnotesize,labelfont={color=DESYorange,footnotesize}]{caption}

\usepackage{xcolor}
\definecolor{light-gray}{gray}{0.8}
\usepackage{relsize}
\renewcommand*{\UrlFont}{\ttfamily\smaller\relax}
\usepackage{hyperref}
\hypersetup{
	colorlinks=true,
	linkcolor=black,
	filecolor=black,
	urlcolor=black,%light-gray,
	breaklinks=true
}



\usepackage[compat=1.1.0]{tikz-feynman}
\usepackage{contour}

\usetikzlibrary{arrows,shapes,positioning}
\usetikzlibrary{decorations.markings}
\usetikzlibrary{decorations.pathmorphing}
\usetikzlibrary{decorations.pathreplacing}
\usetikzlibrary{patterns}
\usetikzlibrary{plotmarks}
\usetikzlibrary{shadows}

\tcbuselibrary{skins}

% \tikzstyle arrowstyle=[scale=1.5]
% \tikzstyle directed=[postaction={decorate,decoration={markings,
%    mark=at position .55 with {\arrow[arrowstyle]{stealth}}}}]
% \tikzstyle reverse directed=[postaction={decorate,decoration={markings,
%    mark=at position .55 with {\arrowreversed[arrowstyle]{stealth};}}}]
% \tikzset{
%    boson/.style={decorate, decoration={snake}, draw=black},
%    fermion/.style={draw=black, postaction={decorate},
%        decoration={markings,mark=at position .6 with {\arrow[arrowstyle]{stealth}}}},
%    fbar/.style={draw=black, postaction={decorate},
%        decoration={markings,mark=at position .6 with {\arrowreversed[arrowstyle]{stealth}}}},
%    gluon/.style={decorate, draw=black,
%        decoration={coil,amplitude=4pt, segment length=5pt}},
% }

\newcommand*\at{{\fontfamily{ptm}\selectfont @}}


%%%%%%%%%%%%%%%%%%%%%%%%%%%%%%%%%%%%%%%%%%%%%%%%%%%%%%%%%%%%%%%%%%%%%%%%%%%%%%%
% Code:

\usepackage{ulem} %for underline, crossing our, etc...

\lstset{
%backgroundcolor=\color{lbcolor},
    tabsize=2,
%   rulecolor=,
    language=[GNU]C++,
    % linewidth=0.9\textwidth,
    xleftmargin=.1\textwidth,
        basicstyle=\scriptsize,
        upquote=true,
        aboveskip={1.5\baselineskip},
        columns=fixed,
        showstringspaces=false,
        extendedchars=false,
        breaklines=true,
        prebreak = \raisebox{0ex}[0ex][0ex]{\ensuremath{\hookleftarrow}},
        frame=single,
        % numbers=left,
        showtabs=false,
        showspaces=false,
        showstringspaces=false,
        identifierstyle=\ttfamily,
        keywordstyle=\color[rgb]{0,0,1},
        commentstyle=\color[rgb]{0.026,0.112,0.095},
        stringstyle=\color[rgb]{0.627,0.126,0.941},
        numberstyle=\color[rgb]{0.205, 0.142, 0.73},
%        \lstdefinestyle{C++}{language=C++,style=numbers}’.
    texcl=true,
}

% \usepackage{filecontent}

\newsavebox\mybox

%%%%%%%%%%%%%%%%%%%%%%%%%%%%%%%%%%%%%%%%%%%%%%%%%%%%%%%%%%%%%%%%%%%%%%%%%%%%%%%


\setlength\footnotemargin{10pt}

\let\beginoldtabular=\tabular
\let\endoldtabular\endtabular
\renewenvironment{tabular}
{
	\def\arraystretch{1.5}
	\beginoldtabular
}
{
	\endoldtabular
}

\newcolumntype{C}{>{\centering\arraybackslash}X}
\let\beginoldtabularx=\tabularx
\let\endoldtabularx\endtabularx
\renewenvironment{tabularx}
{
	\def\arraystretch{1.5}
	\beginoldtabularx
}
{
	\endoldtabularx
}


\usepackage{beamerthemesplit}



% \usetheme{CambridgeUS}
\usefonttheme[onlymath]{serif}
\usecolortheme[named=DESYcyan]{structure}
\useoutertheme{infolines}
% \useoutertheme{default}


\setbeamercolor{palette primary}{bg=white}
\setbeamercolor{section in head/foot}{bg=DESYcyan, fg=white}
\setbeamercolor{frametitle}{fg=DESYcyan,bg=white}
\setbeamerfont{frametitle}{series=\bfseries, size=\LARGE}
\setbeamercolor{framesubtitle}{fg=DESYorange,bg=white}
\setbeamerfont{framesubtitle}{series=\bfseries}
\addtobeamertemplate{headline}{\hypersetup{linkcolor=white}\vspace*{0.7em}}{\vspace*{-0.2em}}
% \setbeamertemplate{headline}{\leavevmode \hypersetup{linkcolor=white} }

\setbeamercolor{item}{fg=DESYorange}
\setbeamercolor{subitem}{fg=DESYcyan}
\setbeamercolor{subsubitem}{fg=DESYgray}

\setbeamertemplate{navigation symbols}{}
\setbeamertemplate{footline}{}
\setbeamertemplate{bibliography item}[text]


% \addtobeamertemplate{footline}{}{
% \begin{textblock*}{1cm}[1,1](12.6cm,-0.1cm)  %(5cm,5cm)
% \includegraphics[height=1cm]{\logopath/DESY-logo.jpg}
% \end{textblock*}
% }

\addtobeamertemplate{footline}{}{
\begin{textblock*}{0.9\textwidth}[-0.1,1](-0.5cm, -0.3cm)  %(5cm,5cm)
\textbf{\textcolor{DESYcyan}{DESY}\textcolor{DESYorange}{.}} ~$|$ \insertshorttitle ~$|$ \insertshortauthor ~$|$ \insertdate ~$|$ \hfill Page \insertframenumber/\inserttotalframenumber
\end{textblock*}
\begin{textblock*}{0.9\textwidth}[-0.1,1](5cm,5cm)
\end{textblock*}
}

 \newenvironment{mydesybox}[2]
 {
 	\begin{minipage}{#1}
	\begin{tcolorbox}[beamer,enhanced, colback=DESYcyan!2!white, colframe=DESYcyan, boxrule=0.5mm, title=#2]
}
{
	\end{tcolorbox}
	\end{minipage}
}

\newenvironment{mydesyinlinebox}[2]
{\begin{tcolorbox}[beamer,enhanced, colback=DESYcyan!2!white, colframe=DESYcyan, boxrule=0.5mm, title=#2]  }
{\end{tcolorbox}}

\tcbset{colframe=DESYcyan, colback=DESYcyan!2!white, colupper=black, fonttitle=\bfseries,nobeforeafter,center title}


\setbeamercolor{title}{fg=DESYcyan}
\setbeamerfont{title}{series=\bfseries, size=}
\setbeamercolor{subtitle}{fg=DESYorange}


\setbeamertemplate{title page}
{
  \vbox{}
\vfill
  \begin{flushleft}
    \begin{beamercolorbox}[sep=8pt, size=\Huge]{title}%,center]{title}
      \usebeamerfont{title}%
			{\Huge\inserttitle}\par%
      \ifx\insertsubtitle\@empty%
      \else%
        \vskip0.25em%
        {\usebeamerfont{subtitle}\usebeamercolor[fg]{subtitle}\insertsubtitle\par}%
      \fi%
    \end{beamercolorbox}%
    \vskip1em\par
    \begin{beamercolorbox}[sep=8pt]{author}%,center
      \usebeamerfont{author}\insertauthor
    \end{beamercolorbox} \vspace{-0.7em}
    \begin{beamercolorbox}[sep=8pt]{institute} %,center
      \usebeamerfont{institute}\normalbaselines\insertinstitute
    \end{beamercolorbox} \vspace{-0.5em}
    \begin{beamercolorbox}[sep=8pt]{date} %,center
      \usebeamerfont{date}\insertdate
    \end{beamercolorbox}
    \vskip 1em
    \vfill

	% \hfill
	\hspace{8pt}
	\begin{minipage}{0.35\textwidth}
		\vspace{0.7cm} \vspace{1em}
		\includegraphics[height=0.7cm]{\logopath/DESY-logo.jpg}
		% \hfill
		% \hfill
		\hspace{0.2em}
    \includegraphics[height=0.7cm]{\logopath/2017_H_Logo_CMYK_untereinander_EN.jpg}
    \hfill
	\end{minipage}
	\hfill
	\begin{minipage}{0.6\textwidth}  \raggedleft
    \hfill
		\includegraphics[height=0.7cm]{\logopath/LCC-logo.jpg}
		\hspace{0.2cm}
		\includegraphics[height=0.7cm]{\logopath/ILC-logo.jpg}
		\hspace{0.2cm}
		\includegraphics[height=0.7cm]{\logopath/ildlogo_Zh-mumuh_sharper_whiteonblue.png}
		\vspace{1em}\\
    \hfill
    \includegraphics[height=0.7cm]{\logopath/university-of-cambridge-logo.png}
	\end{minipage}
\end{flushleft}
  %\vfill
}



\newcommand{\myitem}{\item[\ding{228}]}

\newcommand{\makemytitlepage}{\frame[plain]{\titlepage}}

\newcommand{\BackUp}{\appendix\frame{\begin{center}\begin{Huge}Backup Slides\end{Huge}\end{center}}}





%%%%%%%%%%%%%%%%%%%%%%%%%%%%%%%%%%%%%%%%%%%%%%%%%%%%%%%%%%%%%%%%%%%%%%%%%%%%%%%
% Shadowed Images

\usetikzlibrary{shadows,calc}

% code adapted from https://tex.stackexchange.com/a/11483/3954

% some parameters for customization
\def\shadowshift{3pt,-3pt}
\def\shadowradius{6pt}

\colorlet{innercolor}{black!60}
\colorlet{outercolor}{gray!05}

% this draws a shadow under a rectangle node
\newcommand\drawshadow[1]{
    \begin{pgfonlayer}{shadow}
        \shade[outercolor,inner color=innercolor,outer color=outercolor] ($(#1.south west)+(\shadowshift)+(\shadowradius/2,\shadowradius/2)$) circle (\shadowradius);
        \shade[outercolor,inner color=innercolor,outer color=outercolor] ($(#1.north west)+(\shadowshift)+(\shadowradius/2,-\shadowradius/2)$) circle (\shadowradius);
        \shade[outercolor,inner color=innercolor,outer color=outercolor] ($(#1.south east)+(\shadowshift)+(-\shadowradius/2,\shadowradius/2)$) circle (\shadowradius);
        \shade[outercolor,inner color=innercolor,outer color=outercolor] ($(#1.north east)+(\shadowshift)+(-\shadowradius/2,-\shadowradius/2)$) circle (\shadowradius);
        \shade[top color=innercolor,bottom color=outercolor] ($(#1.south west)+(\shadowshift)+(\shadowradius/2,-\shadowradius/2)$) rectangle ($(#1.south east)+(\shadowshift)+(-\shadowradius/2,\shadowradius/2)$);
        \shade[left color=innercolor,right color=outercolor] ($(#1.south east)+(\shadowshift)+(-\shadowradius/2,\shadowradius/2)$) rectangle ($(#1.north east)+(\shadowshift)+(\shadowradius/2,-\shadowradius/2)$);
        \shade[bottom color=innercolor,top color=outercolor] ($(#1.north west)+(\shadowshift)+(\shadowradius/2,-\shadowradius/2)$) rectangle ($(#1.north east)+(\shadowshift)+(-\shadowradius/2,\shadowradius/2)$);
        \shade[outercolor,right color=innercolor,left color=outercolor] ($(#1.south west)+(\shadowshift)+(-\shadowradius/2,\shadowradius/2)$) rectangle ($(#1.north west)+(\shadowshift)+(\shadowradius/2,-\shadowradius/2)$);
        \filldraw ($(#1.south west)+(\shadowshift)+(\shadowradius/2,\shadowradius/2)$) rectangle ($(#1.north east)+(\shadowshift)-(\shadowradius/2,\shadowradius/2)$);
    \end{pgfonlayer}
}

% create a shadow layer, so that we don't need to worry about overdrawing other things
\pgfdeclarelayer{shadow}
\pgfsetlayers{shadow,main}


\newcommand\shadowimage[2][]{%
\begin{tikzpicture}
\node[anchor=south west,inner sep=0] (image) at (0,0) {\includegraphics[#1]{#2}};
\drawshadow{image}
\end{tikzpicture}}
%%%%%%%%%%%%%%%%%%%%%%%%%%%%%%%%%%%%%%%%%%%%%%%%%%%%%%%%%%%%%%%%%%%%%%%%%%%%%%%
% New symbols


\newcommand{\Lagr}{\mathcal{L}} % Lagrangian-L
\newcommand{\oper}{\mathcal{O}} % Operator symbol

%%%%%%%%%%%%%%%%%%%%%%%%%%%%%%%%%%%%%%%%%%%%%%%%%%%%%%%%%%%%%%%%%%%%%%%%%%%%%%%
% DESY text colors

\newcommand\OrangeBF[1]{\textcolor{DESYorange}{\textbf{#1}}}
\newcommand\CyanBF[1]{\textcolor{DESYcyan}{\textbf{#1}}}

%%%%%%%%%%%%%%%%%%%%%%%%%%%%%%%%%%%%%%%%%%%%%%%%%%%%%%%%%%%%%%%%%%%%%%%%%%%%%%%
% Convenient way to set border around input

\newcommand\scaledinput[2][]{
  \scalebox{#1}{\input{#2}}
}

%%%%%%%%%%%%%%%%%%%%%%%%%%%%%%%%%%%%%%%%%%%%%%%%%%%%%%%%%%%%%%%%%%%%%%%%%%%%%%%
% Rotated text

\usepackage{rotating}

%%%%%%%%%%%%%%%%%%%%%%%%%%%%%%%%%%%%%%%%%%%%%%%%%%%%%%%%%%%%%%%%%%%%%%%%%%%%%%%
% Particles
\newcommand{\eP}{e^{+}}
\newcommand{\eM}{e^{-}}
\newcommand{\lP}{l^{+}}
\newcommand{\lM}{l^{-}}

\newcommand{\fbar}{\bar{f}}
\newcommand{\mubar}{\bar{\mu}}

\newcommand{\eL}{\eM_{L}}
\newcommand{\eR}{\eM_{R}}
\newcommand{\pL}{\eP_{L}}
\newcommand{\pR}{\eP_{R}}

\newcommand{\nue}{\nu_e}
\newcommand{\numu}{\nu_{\mu}}
\newcommand{\nutau}{\nu_{\tau}}
\newcommand{\nul}{\nu_l}

\newcommand{\nubar}{\bar{\nu}}
\newcommand{\nuebar}{\bar{\nue}}
\newcommand{\nulbar}{\bar{\nul}}

\newcommand{\qbar}{\bar{q}}
\newcommand{\qu}{q_{u}}
\newcommand{\qd}{q_{d}}
\newcommand{\qubar}{\bar{q}_{u}}
\newcommand{\qdbar}{\bar{q}_{d}}

\newcommand{\tbar}{\bar{t}}

\newcommand{\vis}{\text{vis}}
\newcommand{\inv}{\text{inv}}

\newcommand{\WP}{W^{+}}
\newcommand{\WM}{W^{-}}
%%%%%%%%%%%%%%%%%%%%%%%%%%%%%%%%%%%%%%%%%%%%%%%%%%%%%%%%%%%%%%%%%%%%%%%%%%%%%%%
% Checkmark

\def\checkmark{\tikz\fill[scale=0.4](0,.35) -- (.25,0) -- (1,.7) -- (.25,.15) -- cycle;}
%%%%%%%%%%%%%%%%%%%%%%%%%%%%%%%%%%%%%%%%%%%%%%%%%%%%%%%%%%%%%%%%%%%%%%%%%%%%%%%
% Allow Latex over included graphic
\usepackage[percent]{overpic}

%%%%%%%%%%%%%%%%%%%%%%%%%%%%%%%%%%%%%%%%%%%%%%%%%%%%%%%%%%%%%%%%%%%%%%%%%%%%%%%
% Package for additional symbols like cross-x
\usepackage{pifont}
\newcommand{\xmark}{\ding{55}}%

%%%%%%%%%%%%%%%%%%%%%%%%%%%%%%%%%%%%%%%%%%%%%%%%%%%%%%%%%%%%%%%%%%%%%%%%%%%%%%%
% Allow line-breaks in tabular cell
\usepackage{makecell}

%%%%%%%%%%%%%%%%%%%%%%%%%%%%%%%%%%%%%%%%%%%%%%%%%%%%%%%%%%%%%%%%%%%%%%%%%%%%%%%


\title[Summer Student Report 2019]{Summer Student Report 2019}
\subtitle{FLC group presentation}
\author[Matthew Koster]{ Matthew Koster \inst{1,2}}% \inst{1,2} \and author2 \inst{2}}
\institute[shortinst]{\inst{1} DESY Hamburg \and %
                      \inst{2} University of Cambridge}
\date{September 2, 2019}

\newcommand{\imagepath}{/Users/mbk/Documents/LATEX/SummerStudentReportTemplate-master/Plots/high_stat}
\newcommand{\feynmanpath}{/Users/mbk/Documents/LATEX/SummerStudentReportTemplate-master/LaTeX/FeynmanDiagrams}

\begin{document}

%WARNING: Needs to be compiled using lualatex (might work w/o, look at Feynman diagrams)


\makemytitlepage

% TODO Something about GitHub & Confluence pages and IDR notes

% ----------------------------------------------------------------------------

\begin{frame}\frametitle{Presentation \OrangeBF{Structure}}
    \begin{itemize}
      \item Motivation
      \item My Processor
      \begin{itemize}
          \item Funciton
          \item Neutrino and ISR Corrections
          \item Angle Extrctions
      \end{itemize}
      \item Efficiencies
      \item Conclusions
      \item Outlook
    \end{itemize}
\end{frame}


% ----------------------------------------------------------------------------

\begin{frame}\frametitle{Motivation \OrangeBF{Jakobs thesis}}
I want to extract some angles and do some efficinecies because UPDATE ME
\end{frame}

% ----------------------------------------------------------------------------

\begin{frame}\frametitle{My Processor \OrangeBF{Overview}}
    \begin{minipage}{0.49\textwidth}
        \begin{itemize}
              \item Register Inputs
              \begin{itemize}
                    \item \CyanBF{IsolatedLeptonTagger} - isolated lepton
                    \item \CyanBF{Fastjet} - quark jets and overlay removal
                    \item \CyanBF{MCParticle} - hard collision particles
              \end{itemize}
              \item Analyses Reconstructed particles extracting W bosons 4-momenta
              \item Analyses MC collection extracting angles
              \item Outputs a root file with various relevant variables
       \end{itemize}
    \end{minipage}\hfill
    \begin{minipage}{0.49\textwidth}
        \includegraphics[width=\textwidth]{\imagepath/../plots/new/ROOT.pdf}
    \end{minipage}
\end{frame}

% ----------------------------------------------------------------------------

\begin{frame}\frametitle{My Processor \OrangeBF{Neutrino and ISR Corrections}}

        \only<1>{%
            \begin{minipage}{0.49\textwidth}
                \CyanBF{The system}
                \begin{itemize}\setlength\itemsep{1em}
                        \item \CyanBF{Visible 4-momenta} ${p}^{\mu} = ( E,  {p}_{x}, {p}_{y}, {p}_{z})$
                        \item \CyanBF{Neutrino 4-momenta} ${p}_{\nu}^{\mu} = ( {E}_{\nu},  {p}_{x, \nu}, {p}_{y, \nu}, {p}_{z, \nu})$
                        \item \CyanBF{ISR Photon 4-momenta} ${p}_{\nu}^{\mu} = ( {E}_{\gamma}, 0, 0, {p}_{\gamma})$
                \end{itemize}
                c.f The unconventional ordering of the 4-momenta is because that is how TLorentzVector handels 4-vectors
            \end{minipage}\hfill
            \begin{minipage}{0.49\textwidth}
                \resizebox{0.95\linewidth}{!}{%
                \centering
                \begin{tikzpicture}
  \begin{feynman}[scale=1] % using the vertex in brackets () allows fixing of vertex
    \vertex (m) at (-1, 0);
    \vertex (n) at (0.5, 0);
    \vertex (a) at (-1.5,-1);s
    \vertex (b) at ( 1.75,-1) ;
    \vertex (c) at (-1.5, 1);
    \vertex (d) at ( 1.75, 1) ;
    \vertex (u) at ( 3.5,-1.5) {$\mu^-$};
    \vertex (v) at ( 3.5,-0.5) {$\bar{\nu}_\mu$};
    \vertex (q1) at ( 3.5, 1.5) {$q$};
    \vertex (q2) at ( 3.5, 0.5) {$\bar{q'}$};
    \vertex (i1) at (-3,-1) {$e^-$};
    \vertex (i2) at (-3, 1) {$e^+$};
    \diagram* {
      (i1) -- [fermion] (m) ,
      (i2) -- [anti fermion] (m) ,
      (m) -- [photon, edge label=$Z/\gamma$] (n),
      (b) -- [photon, edge label=$W^-$, swap] (n),
      (d) -- [photon, edge label=$W^+$] (n),
      (b) -- [fermion] (u),
      (b) -- [anti fermion] (v),
      (d) -- [fermion] (q1),
      (d) -- [anti fermion] (q2),
    };
  \end{feynman}
\end{tikzpicture}

                }
                \\
                \centering
                \begin{tikzpicture}
  \begin{feynman}[scale=1] % using the vertex in brackets () allows fixing of vertex
    \vertex (n) at (0, 0.75) ;
    \vertex (m) at (0, -0.75) ;
    \vertex (a) at (-1,-1) ;
    \vertex (b) at ( 1.25,-1) ;
    \vertex (c) at (-1, 1);
    \vertex (d) at ( 1.25, 1) ;
    \vertex (u) at ( 3,-1.5) {$l^-$};
    \vertex (v) at ( 3,-0.5) {$\bar{\nu}_l$};
    \vertex (q1) at ( 3, 1.5) {$q$};
    \vertex (q2) at ( 3, 0.5) {$\bar{q'}$};
    \vertex (i1) at (-2.5,-1) {$e^-$};
    \vertex (i2) at (-2.5, 1) {$e^+$};
    \diagram* {
      (i1) -- [fermion] (m) -- [fermion, edge label=$\nu_e$] (n),
      (i2) -- [anti fermion] (n),
      (b) -- [photon, edge label=$W^-$, swap] (m),
      (d) -- [photon, edge label=$W^+$] (n),
      (b) -- [fermion] (u),
      (b) -- [anti fermion] (v),
      (d) -- [fermion] (q1),
      (d) -- [anti fermion] (q2),
    };
  \end{feynman}
\end{tikzpicture}

            \end{minipage}

          }\only<2>{%
                 Consider only energy and momentum conservation, where the invarient mass of the neutrino and ISR photon is zero.\\
                 \CyanBF{Simple energy equation} (I. Marchesini ***CITE***)
                 \begin{equation}
                     {E}_{\gamma} = \frac{ {(500 - E)}^2 - {p}^{2}}{1000 -2 E  \mp 2{p}_{z}}
                 \end{equation}

          }\only<3>{%
                 \begin{itemize}
                 \item Negative energies arise!
                 \item It often boils down to negative invisible invarient mass
                 \item This is because of Reconstruction
                 \item Handel carefully in code
                 \item Perhaps energy assumption is invalid
                 \item Perhaps zero invarient mass of photon is invalid
                 \item What else can we check?
                \end{itemize}

          }\only<4>{%
                 \begin{minipage}{0.49\textwidth}
                     \begin{itemize}
                         \item The ${e}^{-}{e}^{+}$ collision is \CyanBF{not in the center of mass frame}, the inital state has a 4-momentum of,
                         \begin{equation}
                             {p}^{\mu} = ( 500 \sin{(\frac{0.014}{2})}, 0, 0, 500)\, GeV.
                         \end{equation}
                         \item \CyanBF{Lorentz Boost} into center of mass frame to conduct calculations
                         \item Improvement
                    \end{itemize}
                 \end{minipage}\hfill
                 \begin{minipage}{0.49\textwidth}
                     \includegraphics[width=\textwidth]{\imagepath/Boost.pdf}
                 \end{minipage}

         }\only<5>{%
                \begin{minipage}{0.49\textwidth}
                    \begin{itemize}
                        \item Perhaps the overlay removal processor is not performing properly
                        \item Try \CyanBF{Cheat Overlay} using TJJetOverlayRemoval (Jakob ***CITE***)
                        \item Improves ${m}_{W}^{had}$ as expected
                        \item Slightly worsens ${m}_{W}^{lep}$ \\
                        \, \rightarrow \, statistical fluctuation?
                        \item ${m}_{W}^{lep}$ is \CyanBF{not particularly sensitive} to it due to the complicated nature of the ${E}_{\gamma}$ formula
                    \end{itemize}
                \end{minipage}\hfill
                \begin{minipage}{0.49\textwidth}
                    \includegraphics[width=0.8\textwidth]{\imagepath/Jet.pdf}
                    \includegraphics[width=0.8\textwidth]{\imagepath/Lep.pdf}
                \end{minipage}

          }\only<6>{%
            Consider only energy and momentum conservation, where the invarient mass of the neutrino and ISR photon is nolonger assumed zero.\\
            \CyanBF{Full energy equation}
            \begin{equation}
                {E}_{\gamma}    = \frac{{\lambda}(500 - E)  \pm {p}_{z}\sqrt{ {\lambda}^{2} - [{(500 - E)}^{2} -{p}_{z}^{2}]{m}_{\gamma}^{2}}}{{(500 - E)}^{2} -   {p}_{z}^{2}}
            \end{equation}
            Where for convenience I have defined \CyanBF{lambda},
            \begin{equation}
                {\lambda} = \frac{1}{2}[{(500 - E)}^2 - {p}^{2} + {m}_{\gamma}^{2} - {m}_{\nu}^{2}] \, .
            \end{equation}

          }\only<7>{%
              \begin{minipage}{0.49\textwidth}
                  \begin{itemize}
                      \item Using this formula with ${m}_{\nu} = 0$ and ${m}_{\gamma}$ extracted from the MonteCarlo collection
                      \item At low statistics there appeard to be a difference but at high statistics it was seen to be negligable
                      \item The reconstruction is not sensitive to the ISR invairent mass
                  \end{itemize}
              \end{minipage}\hfill
              \begin{minipage}{0.49\textwidth}
                  \includegraphics[width=\textwidth]{\imagepath/Mass.pdf}
              \end{minipage}

          }\only<8>{%
            \begin{itemize}
                \item Add a third option of there being no ISR photon such that  ${E}_{\gamma} = 0$
                \item When this option is chosen, the \CyanBF{formula struggles to reconstruct small ${E}_{\gamma}$ values}, so it is an improvement.
            \end{itemize}
                \includegraphics[width=0.45\textwidth]{\imagepath/Photon.pdf}
                \includegraphics[width=0.45\textwidth]{\imagepath/Photon_zoom.pdf}

            }\only<9>{%
            \begin{minipage}{0.49\textwidth}
                \begin{itemize}
                    \item The ${E}_{\gamma}$ forumula is an improvement on the solution that neglects ISR
                    \item Adding a solution for no ISR improves the estimate again
                \end{itemize}
            \end{minipage}\hfill
            \begin{minipage}{0.49\textwidth}
                \includegraphics[width=\textwidth]{\imagepath/Mass3.pdf}
            \end{minipage}

              }
\end{frame}

% ----------------------------------------------------------------------------

\begin{frame}\frametitle{My Processor \OrangeBF{Angle Extractions}}

        \only<1>{%
        From the MC collection I extracted the appropriate angles (${\theta}_{W^{-}}, {\theta}_{l}^{*}, {\phi}_{l}^{*} $) for Jakob as defined by R.Karl ***CITE*** slightly edited such that we boost into the ${W}^{lep}$ frame not the ${W}^{-}$
        \includegraphics[width=0.8\textwidth]{\imagepath/AngleDiag.pdf}

        }\only<2>{%
        \begin{minipage}{0.32\textwidth}
        \includegraphics[width=\textwidth]{\imagepath/ThetaMin_cos.pdf}
        \end{minipage}\hfill
        \begin{minipage}{0.32\textwidth}
        \includegraphics[width=\textwidth]{\imagepath/ThetaLep_cos.pdf}
        \end{minipage}
        \begin{minipage}{0.32\textwidth}
        \includegraphics[width=\textwidth]{\imagepath/PhiLep.pdf}
        \end{minipage}
        }

\end{frame}

% ----------------------------------------------------------------------------

% ----------------------------------------------------------------------------

\begin{frame}\frametitle{Efficiencies \OrangeBF{Applying cuts}}
      \only<1>{%
      \begin{minipage}{0.49\textwidth}

          \begin{table}[!]
            \centering
            \caption{
              Selection efficiency of sequantially applied cuts. Where the post ISR correction ${m}_{W}^{lep}$ was calculated using all 3 possible ${E}_{\gamma}$ solutions. (*) Means my and Ivan's cuts differ slightly
            }
                \resizebox{0.8\textwidth}{!}{%
                \begin{tabular}{|l|l|l|l|l|l|} \hline
                  Order & Cut description & \multicolumn{4}{c|}{Efficiency [\%]} \\ \cline{3-6}
                  & & \multicolumn{3}{c|}{My Results} & Ivan's Results \\  \cline{3-5}
                  & & n = 2129 & \multicolumn{2}{c|}{n = 99419} & n = 107233 \\ \cline{4-5}
                  & & & no cheat & cheat & \\ \hline \hline
                  0 & muon signal & 100.00 & 100.00 & 100.00 & 100.00 \\ \hline
                  1 & track multiplicity ${n}_{tracks} \ge 10$ & 97.13 & 97.01 & 96.23 & 99.996 \\ \hline
                  2 & center of mass energy $\sqrt{s} > 100$ GeV & 92.29 & 91.69 & 84.35 & 97.96 \\ \hline
                  3 & total transverse momentum ${P}_{T} > 5$ GeV & 91.16 & 90.47 & 83.28 & 96.69 \\ \hline
                  4 & total energy ${E}_{SUM} < 500$ GeV & 89.66 & 89.28 & 82.70 & 95.36 \\ \hline
                  5 & $\ln{({y}_{+})} \in [-12, -3]$ (*) & 88.69 & 88.08 & 82.47 & 95.01 \\ \hline
                  6 & 1 lepton found (*) & 80.65 & 80.77 & 81.50 & 78.75 \\ \hline
                  7 & pre ISR correction ${m}_{W}^{lep} \in [20, 250]$ GeV &  78.23 & 77.94 & 77.84 & 76.61 \\ \hline
                  8 & tau discrimination &  76.05 & 75.60 & 75.73 & 74.07 \\ \hline
                  9 & charged lepton (*) & 76.05 & 75.60 & 75.73 & 73.51 \\ \hline
                  10 & isolation variable ${\Delta\Omega}_{iso} > 0.5$ & 76.01 & 75.58 & 75.72 & 73.42 \\ \hline
                  11 & post ISR correction ${m}_{W}^{lep} \in [40, 120]$ GeV & 72.90 & 72.77 & 72.33 & 70.13 \\ \hline
                  12 & post ISR correction ${m}_{W}^{had} \in [40, 120]$ GeV & 63.21 & 62.92 & 70.52 & 66.93 \\ \hline
                  13 & $\cos{{\theta}_{W}} > -0.95$ & 63.02 & 62.65 & 70.21 & 66.78 \\ \hline
                \end{tabular}
                }
            \end{table}
            \end{minipage}\hfill
            \begin{minipage}{0.49\textwidth}
                \includegraphics[width=\textwidth]{\imagepath/Histflow3.pdf}
            \end{minipage}

        }\only<2>{%
            \begin{itemize}
                \item track mulitplicity was taken as the number of reconstructed charged particles.
                \item $\Delta{\Omega}_{iso}$ defined as,
                    \begin{align}
                         ({\phi}_{lep} - {\phi}_{had}) < \pi \to \Delta{\Omega}_{iso} &= \sqrt{{({\theta}_{lep} - {\theta}_{had})}^{2}+{({\phi}_{lep} - {\phi}_{had})}^{2}} \\
                         ({\phi}_{lep} - {\phi}_{had}) \ge \pi \to \Delta{\Omega}_{iso} &= \sqrt{{({\theta}_{lep} - {\theta}_{had})}^{2} + {(2\pi - |{\phi}_{lep} - {\phi}_{had} |)}^{2}} \, .
                    \end{align}
                \item ${\tau}_{discr}$ defined by
                    \begin{align}
                        {\tau}_{discr} = {(\frac{2{E}_{lep}}{\sqrt{s}})}^{2} + {(\frac{{m}_{W}^{lep}}{{m}_{W}^{true}})}^{2}
                    \end{align}
            \end{itemize}

        }\only<3>{%
            The selection efficiencies of the extracted angles after applying all the previous cuts (binomal errors), with the angular distribution below for reference\\
            \begin{minipage}{0.32\textwidth}
            \includegraphics[width=\textwidth]{\imagepath/E_ThetaMin_cos_err.pdf}
            \includegraphics[width=\textwidth]{\imagepath/ThetaMin_cos.pdf}
            \end{minipage}\hfill
            \begin{minipage}{0.32\textwidth}
            \includegraphics[width=\textwidth]{\imagepath/E_ThetaLep_cos_err.pdf}
            \includegraphics[width=\textwidth]{\imagepath/ThetaLep_cos.pdf}
            \end{minipage}
            \begin{minipage}{0.32\textwidth}
            \includegraphics[width=\textwidth]{\imagepath/E_PhiLep_Err.pdf}
            \includegraphics[width=\textwidth]{\imagepath/PhiLep.pdf}
            \end{minipage}
        }

\end{frame}

% ----------------------------------------------------------------------------


\begin{frame}\frametitle{Conclusions \OrangeBF{do tacheyons exist?}}

\end{frame}

% ----------------------------------------------------------------------------

\begin{frame}\frametitle{Outlook \OrangeBF{do things}}

\end{frame}

% ----------------------------------------------------------------------------
\begin{frame}\frametitle{Back Up Slides \OrangeBF{Low Statistics}}
    \only<1>{%
        \begin{minipage}{0.49\textwidth}
            \centering
            \includegraphics[width=0.8\textwidth]{\imagepath/../low_stat/Boost.pdf}
            \includegraphics[width=0.8\textwidth]{\imagepath/../low_stat/Mass.pdf}
        \end{minipage}
        \begin{minipage}{0.49\textwidth}
            \centering
            \includegraphics[width=0.8\textwidth]{\imagepath/../low_stat/Jet.pdf}
            \includegraphics[width=0.8\textwidth]{\imagepath/../low_stat/Lep.pdf}
        \end{minipage}

    }\only<2>{%
        \centering
        \includegraphics[width=0.5\textwidth]{\imagepath/../low_stat/Mass3.pdf}

    }\only<3>{%
        \includegraphics[width=0.49\textwidth]{\imagepath/../low_stat/Photon.pdf}
        \includegraphics[width=0.49\textwidth]{\imagepath/../low_stat/Photon_zoom.pdf}
    }
    \end{frame}

    % ----------------------------------------------------------------------------
    \begin{frame}\frametitle{Back Up Slides \OrangeBF{Log Plots}}

        \includegraphics[width=0.49\textwidth]{\imagepath/Mass3_log.pdf}
        \includegraphics[width=0.49\textwidth]{\imagepath/Mass_log.pdf}

    \end{frame}

    % ----------------------------------------------------------------------------
    \begin{frame}\frametitle{Back Up Slides \OrangeBF{${E}_{\gamma} < 1$ Efficiencies}}

    \only<1>{%
        \begin{minipage}{0.49\textwidth}
            \centering
            \includegraphics[width=\textwidth]{\imagepath/P_HistFlow3.pdf}
        \end{minipage}
        \begin{minipage}{0.49\textwidth}
            \centering
            \includegraphics[width=\textwidth]{\imagepath/P_HistFlow3_All.pdf}
        \end{minipage}

    }\only<2>{%
        \begin{minipage}{0.32\textwidth}
            \centering
            \includegraphics[width=\textwidth]{\imagepath/P_E_ThetaMin_cos_err.pdf}
        \end{minipage}
        \begin{minipage}{0.32\textwidth}
            \centering
            \includegraphics[width=\textwidth]{\imagepath/P_E_ThetaLep_cos_err.pdf}
        \end{minipage}
        \begin{minipage}{0.32\textwidth}
            \centering
            \includegraphics[width=\textwidth]{\imagepath/P_E_PhiLep_err.pdf}
        \end{minipage}
        \begin{minipage}{0.32\textwidth}
            \centering
            \includegraphics[width=\textwidth]{\imagepath/P_E_ThetaMin_cos_err_Both.pdf}
        \end{minipage}
        \begin{minipage}{0.32\textwidth}
            \centering
            \includegraphics[width=\textwidth]{\imagepath/P_E_ThetaLep_cos_err_Both.pdf}
        \end{minipage}
        \begin{minipage}{0.32\textwidth}
            \centering
            \includegraphics[width=\textwidth]{\imagepath/P_E_PhiLep_err_Both.pdf}
        \end{minipage}
    }

\end{frame}

% ----------------------------------------------------------------------------

\end{document}
